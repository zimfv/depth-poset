\documentclass{article}
\usepackage{amsfonts} 
\usepackage{amsmath}
\usepackage{hyperref}
\usepackage{graphicx}
\usepackage{booktabs}
\usepackage{makecell}

\begin{document}

\section{Model}
\subsection{Complex, Filtration and Homotopy}
\par In this model we define the simplicial complex by the Delauney triangulation of $n = 10$ points uniformly distributed in $[0, 1]^d$ for $d = 2$.
\par We defining the filtration on this complex, by assuming uniformly distributed in $[0, 1]$ height $h(f)$ for each vertex $v$. Then the filtration value of the simplex will be the maximum haight of its vertices.
$$
    f(\sigma) = \max_{v\in \sigma} h(v)
$$
\par We define 2 filtrations like this and study the linear homotopy between them. In the Figure \ref{fig:complex} we can see these 2 filtrations:
\begin{figure}[h!]
    \centering
    \includegraphics[width=1.2\textwidth]{pics/heights linear homotopy - complex with filtrations.png}
    \caption{2 filtrations on the defined complex.}
    \label{fig:complex}
\end{figure}

\par Having these 2 filtrations we can define the homotopy between them by defining the linear homotopy between heights:
$$
    h_t(v) = h_0(v)\cdot(1 - t) + h_1(v)\cdot t
$$
$$
    f_t(\sigma) = \max_{v\in \sigma} h_t(v)
$$

\subsection{Transpositions}
\par In the Figure \ref{fig:homotopy} we can see the vertices height $h_t(v)$ during this homotopy. And the transpositions will happen, when these lines cross.
\begin{figure}[h!]
    \centering
    \includegraphics[width=1.2\textwidth]{pics/heights linear homotopy - heights homotopy.png}
    \caption{Heights of Vertices during the Homotopy.}
    \label{fig:homotopy}
\end{figure}
\par ...

\end{document}