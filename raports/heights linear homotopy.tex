\documentclass{article}
\usepackage{amsfonts} 
\usepackage{amsmath}
\usepackage{hyperref}
\usepackage{graphicx}
\usepackage{booktabs}
\usepackage{makecell}

\begin{document}

\section{Model}
\subsection{Complex, Filtration and Homotopy}
\par In this model we define the simplicial complex by the Delauney triangulation of $n = 10$ points uniformly distributed in $[0, 1]^d$ for $d = 2$.
\par We defining the filtration on this complex, by assuming uniformly distributed in $[0, 1]$ height $h(f)$ for each vertex $v$. Then the filtration value of the simplex will be the maximum haight of its vertices.
$$
    f(\sigma) = \max_{v\in \sigma} h(v)
$$
\par We define 2 filtrations like this and study the linear homotopy between them. In the Figure \ref{fig:complex} we can see these 2 filtrations:
\begin{figure}[h!]
    \centering
    \includegraphics[width=1.2\textwidth]{pics/heights linear homotopy - complex with filtrations.png}
    \caption{2 filtrations on the defined complex.}
    \label{fig:complex}
\end{figure}

\par Having these 2 filtrations we can define the homotopy between them by defining the linear homotopy between heights:
$$
    h_t(v) = h_0(v)\cdot(1 - t) + h_1(v)\cdot t
$$
$$
    f_t(\sigma) = \max_{v\in \sigma} h_t(v)
$$

\subsection{Transpositions}
\par In the Figure \ref{fig:homotopy} we can see the vertices height $h_t(v)$ during this homotopy. 
\begin{figure}[h!]
    \centering
    \includegraphics[width=1.2\textwidth]{pics/heights linear homotopy - heights homotopy.png}
    \caption{Heights of Vertices during the Homotopy.}
    \label{fig:homotopy}
\end{figure}
\par When there is a cross of lines $h_t(i)$ and $h_t(j)$ ($t: h_t(i) = h_t(j)$) there is transposition of heights of vertices $i$ and $j$. This means that happens reordering in the filtration $f_t$. The order given by $f_{t - \varepsilon}$ changes to the order given by $f_{t + \varepsilon}$.
\par Let's $h_t(i) < h_t(j)$. We can define 3 groups of simplices moved in the order:
\begin{enumerate}
    \item $A = \{\sigma: i \in \sigma, j\notin \sigma, \not\exists v\in \sigma: h(v) > h(j)\}$
    \item $B = \{\sigma: i \notin \sigma, j\in \sigma, \not\exists v\in \sigma: h(v) > h(j)\}$
    \item $C = \{\sigma: i \in \sigma, j\in \sigma, \not\exists v\in \sigma: h(v) > h(j)\}$
\end{enumerate}

In the order given by $f_{t - \varepsilon}$ the group $A$ stays on the first $\#A$ places, and in the order given by $f_{t + \varepsilon}$ the group $B$ stays on the first $\#B$ places.
\par There are many paths of transpositions in the order, which brings us from the order $f_{t - \varepsilon}$ to the order $f_{t + \varepsilon}$ with the conctrain that $\sigma_0$ stays before $\sigma_1$ if $\sigma_0 \subset \sigma_1$. We difined 2 of them:
\begin{itemize}
    \item[Up directed] The first we move simplices of group $B$ to the first places, and then we move simplices to group $C$ to their places in $f_{t + \varepsilon}$.
    \item[Down directed] The first we move simplices of group $C$ to the last places, and then we move simplices of group $A$ to their places in $f_{t + \varepsilon}$.
\end{itemize}

\end{document}